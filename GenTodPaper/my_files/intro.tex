\section{Introductions}
\begin{itemize}
    \item Basic Tod intro
    \item Challenge of zero-shot TOD
    \item Existing methods
    \item Our Method
    \item Pros of our methodology
    \item Summary of results
\end{itemize}

TOD systems interact with users in the form of dialog using natural language, to accomplish user tasks.
The system needs to understand user needs and provide the best possible response to the user.
The task of extracting user intent and goals from conversations by filling belief slots is called Dialog State Tracking (DST)~\cite{wang-etal-2016-inner}.
Using the DST and dialog history, the system needs to decide what actions to take and then convey that action in the form of natural language to the user.

Traditional TOD systems were built using a pipeline approach, where each component was created separately and then integrated together.
However, with the adaptation of large pretrained language models~\cite{Devlin2019BERTPO,Radford2019LanguageMA},
researchers have moved towards end-to-end systems~\cite{HosseiniAsl2020ASL,Peng2021SoloistBT,Lee2020SUMBTLaRLEN,Yang2020UBARTF,Jeon2021DORATP,Sun2022BORTBA,Yang2022UBARv2TM,Noroozi2020AFA}.,
and have formulated the problem as a cascaded generation problem~\cite{su2021multi}, which is to sequentially generate the DST, system action and response.

